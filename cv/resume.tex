%%%%%%%%%%%%%%%%%%%%%%%%%%%%%%%%%%%%%%%%%%%%%%%%%%%%%%%%%%%%%%%%%%%%%%%%
%%%%%%%%%%%%%%%%%%%%%% Simple LaTeX CV Template %%%%%%%%%%%%%%%%%%%%%%%%
%%%%%%%%%%%%%%%%%%%%%%%%%%%%%%%%%%%%%%%%%%%%%%%%%%%%%%%%%%%%%%%%%%%%%%%%

%%%%%%%%%%%%%%%%%%%%%%%%%%%%%%%%%%%%%%%%%%%%%%%%%%%%%%%%%%%%%%%%%%%%%%%%
%% NOTE: If you find that it says                                     %%
%%                                                                    %%
%%                           1 of ??                                  %%
%%                                                                    %%
%% at the bottom of your first page, this means that the AUX file     %%
%% was not available when you ran LaTeX on this source. Simply RERUN  %%
%% LaTeX to get the ``??'' replaced with the number of the last page  %%
%% of the document. The AUX file will be generated on the first run   %%
%% of LaTeX and used on the second run to fill in all of the          %%
%% references.                                                        %%
%%%%%%%%%%%%%%%%%%%%%%%%%%%%%%%%%%%%%%%%%%%%%%%%%%%%%%%%%%%%%%%%%%%%%%%%

%%%%%%%%%%%%%%%%%%%%%%%%%%%% Document Setup %%%%%%%%%%%%%%%%%%%%%%%%%%%%

% Don't like 10pt? Try 11pt or 12pt
\documentclass[10pt]{article}

% This is a helpful package that puts math inside length specifications
\usepackage{calc}
\usepackage{pifont}
\usepackage{marvosym}


% Simpler bibsection for CV sections
% (thanks to natbib for inspiration)
\makeatletter
\newlength{\bibhang}
\setlength{\bibhang}{1em}
\newlength{\bibsep}
 {\@listi \global\bibsep\itemsep \global\advance\bibsep by\parsep}
\newenvironment{bibsection}%
        {\vspace{-\baselineskip}\begin{list}{}{%
       \setlength{\leftmargin}{\bibhang}%
       \setlength{\itemindent}{-\leftmargin}%
       \setlength{\itemsep}{\bibsep}%
       \setlength{\parsep}{\z@}%
        \setlength{\partopsep}{0pt}%
        \setlength{\topsep}{0pt}}}
        {\end{list}\vspace{-.6\baselineskip}}
\makeatother

% Layout: Puts the section titles on left side of page
\reversemarginpar

%
%         PAPER SIZE, PAGE NUMBER, AND DOCUMENT LAYOUT NOTES:
%
% The next \usepackage line changes the layout for CV style section
% headings as marginal notes. It also sets up the paper size as either
% letter or A4. By default, letter was used. If A4 paper is desired,
% comment out the letterpaper lines and uncomment the a4paper lines.
%
% As you can see, the margin widths and section title widths can be
% easily adjusted.
%
% ALSO: Notice that the includefoot option can be commented OUT in order
% to put the PAGE NUMBER *IN* the bottom margin. This will make the
% effective text area larger.
%
% IF YOU WISH TO REMOVE THE ``of LASTPAGE'' next to each page number,
% see the note about the +LP and -LP lines below. Comment out the +LP
% and uncomment the -LP.
%
% IF YOU WISH TO REMOVE PAGE NUMBERS, be sure that the includefoot line
% is uncommented and ALSO uncomment the \pagestyle{empty} a few lines
% below.
%

%% Use these lines for letter-sized paper
%\usepackage[paper=letterpaper,
%           %includefoot, % Uncomment to put page number above margin
%            marginparwidth=0.7in,     % Length of section titles
%            marginparsep=.05in,       % Space between titles and text
%            margin=0.5in,               % 1 inch margins
%            includemp]{geometry}

% Use these lines for A4-sized paper
\usepackage[paper=a4paper,
            %includefoot, % Uncomment to put page number above margin
            marginparwidth=24mm,    % Length of section titles
            marginparsep=1mm,       % Space between titles and text
            margin=15mm,              % 25mm margins
            includemp]{geometry}

%% More layout: Get rid of indenting throughout entire document
\setlength{\parindent}{0in}

%% This gives us fun enumeration environments. compactitem will be nice.
\usepackage{paralist}
\usepackage[shortlabels]{enumitem}
\usepackage{fontawesome}
% \usepackage[misc]{ifsym}
%% Reference the last page in the page number
%
% NOTE: comment the +LP line and uncomment the -LP line to have page
%       numbers without the ``of ##'' last page reference)
%
% NOTE: uncomment the \pagestyle{empty} line to get rid of all page
%       numbers (make sure includefoot is commented out above)
%
\usepackage{fancyhdr,lastpage}
\pagestyle{fancy}
%\pagestyle{empty}      % Uncomment this to get rid of page numbers
\fancyhf{}\renewcommand{\headrulewidth}{0pt}
\fancyfootoffset{\marginparsep+\marginparwidth}
\newlength{\footpageshift}
\setlength{\footpageshift}
          {0.1\textwidth+0.1\marginparsep+0.1\marginparwidth-2in}
\lfoot{\hspace{\footpageshift}%
       \parbox{3.5in}{\, \hfill %
                    \arabic{page} of \protect\pageref*{LastPage} % +LP
%                    \arabic{page}                               % -LP
                    \hfill \,}}

% Finally, give us PDF bookmarks
\usepackage{color,hyperref}
\definecolor{darkblue}{rgb}{0.0,0.0,0.3}
\hypersetup{colorlinks,breaklinks,
            linkcolor=darkblue,urlcolor=darkblue,
            anchorcolor=darkblue,citecolor=darkblue}

%%%%%%%%%%%%%%%%%%%%%%%% End Document Setup %%%%%%%%%%%%%%%%%%%%%%%%%%%%


%%%%%%%%%%%%%%%%%%%%%%%%%%% Helper Commands %%%%%%%%%%%%%%%%%%%%%%%%%%%%

% The title (name) with a horizontal rule under it
%
% Usage: \makeheading{name}
%
% Place at top of document. It should be the first thing.
\newcommand{\makeheading}[1]%
        {\hspace*{-\marginparsep minus \marginparwidth}%
         \begin{minipage}[t]{\textwidth+\marginparwidth+\marginparsep}%
                {\large \bfseries #1}\\[-0.15\baselineskip]%
                 \rule{\columnwidth}{1pt}%
         \end{minipage}}

% The section headings
%
% Usage: \section{section name}
%
% Follow this section IMMEDIATELY with the first line of the section
% text. Do not put whitespace in between. That is, do this:
%
%       \section{My Information}
%       Here is my information.
%
% and NOT this:
%
%       \section{My Information}
%
%       Here is my information.
%
% Otherwise the top of the section header will not line up with the top
% of the section. Of course, using a single comment character (%) on
% empty lines allows for the function of the first example with the
% readability of the second example.
\renewcommand{\section}[2]%
        {\pagebreak[2]\vspace{1\baselineskip}%
         \phantomsection\addcontentsline{toc}{section}{#1}%
         \hspace{0in}%
         \marginpar{
         \raggedright \scshape #1}#2}

% An itemize-style list with lots of space between items
\newenvironment{outerlist}[1][\enskip\textbullet]%
        {\begin{itemize}[#1]}{\end{itemize}%
         \vspace{-0.6\baselineskip}}

% An environment IDENTICAL to outerlist that has better pre-list spacing
% when used as the first thing in a \section
\newenvironment{lonelist}[1][\enskip\textbullet]%
        {\vspace{-\baselineskip}\begin{list}{#1}{%
        \setlength{\partopsep}{0pt}%
        \setlength{\topsep}{0pt}}}
        {\end{list}\vspace{-.6\baselineskip}}

% An itemize-style list with little space between items
% \newenvironment{innerlist}[1][\enskip\textbullet]%
\newenvironment{innerlist}[1][\enskip$\circ$]%
        {\begin{compactitem}[#1]}{\end{compactitem}}

% An environment IDENTICAL to innerlist that has better pre-list spacing
% when used as the first thing in a \section
\newenvironment{loneinnerlist}[1][\enskip\textbullet]%
        {\vspace{-\baselineskip}\begin{compactitem}[#1]}
        {\end{compactitem}\vspace{-.6\baselineskip}}

% To add some paragraph space between lines.
% This also tells LaTeX to preferably break a page on one of these gaps
% if there is a needed pagebreak nearby.
\newcommand{\blankline}{\quad\pagebreak[2]}

% Uses hyperref to link DOI
\newcommand\doilink[1]{\href{http://dx.doi.org/#1}{#1}}
\newcommand\doi[1]{doi:\doilink{#1}}


%%%%%%%%%%%%%%%%%%%%%%%% End Helper Commands %%%%%%%%%%%%%%%%%%%%%%%%%%%

%%%%%%%%%%%%%%%%%%%%%%%%% Begin CV Document %%%%%%%%%%%%%%%%%%%%%%%%%%%%

%\hyphenpenalty = 9999
\def\vs{\vspace{-0.1in}}
\begin{document}
% \makeheading{Curriculum Vitae\\ [0.3cm] TIEP HUU VU\quad~~~~~~\quad\quad\quad\quad\quad\quad\quad\quad\quad\quad\quad\quad\quad\quad{\small Last update: December 17, 2015}}
\makeheading{Quan Meng 孟权 \hfill {\small Last update: \today}}


\section{Contact Information}

\newlength{\rcollength}\setlength{\rcollength}{3 in}
\vs
\begin{tabular}[t]{@{}p{\textwidth-\rcollength}p{\rcollength}}
\texttt{Address}: Pudong, Shanghai, China & \\
{\faWechat} \texttt{Wechat}: nenhabkks   & {\faMousePointer} \texttt{Homepage:}\href{https://mq66.github.io}{https://mq66.github.io}\\
{\large\faMobile} \texttt{Tel:} +86 19821254220     &  {\faEnvelopeO} \texttt{E-mail:}\href{mailto:mengquan@shanghaitech.edu.cn}{mengquan@shanghaitech.edu.cn}\\
\end{tabular}


%% ========= Academic History ==============================
\section{Academic History}
\href{https://www.shanghaitech.edu.cn/eng/}{\textbf{ShanghaiTech University}} \hfill Fall 2019 - Spring 2022 (expected)
\begin{outerlist}
	\item M.S. in Computer Science and Engineering
	\item Advisor: Prof. \href{http://www.yu-jingyi.com/cv/}{Jingyi Yu}
\end{outerlist}

\vspace{0.1in}
\href{https://www.en.sdu.edu.cn}{\textbf{Shandong University}} \hfill Fall 2015 - Spring 2019
\begin{outerlist}
	\item B.S. in Automatic Control % GPA: 4.2/5 
	\item Advisor: Prof. \href{https://faculty.sdu.edu.cn/liuguoliang1/en/index.htm}{Guoliang Liu}
\end{outerlist}


%% ========= Publications =================================
\section{Publications}
\vspace{-.25in}
\begin{enumerate} 
	\item {\bf Quan Meng}, Anpei Chen, Haimin Luo, Minye Wu, Hao Su, Lan Xu, Xuming He, and Jingyi Yu \\
	GNeRF: GAN-Based Neural Radiance Field without Posed Camera \\
	\textit{Proceedings of the IEEE/CVF International Conference on Computer Vision (ICCV), 2021} \\
	{\bf Oral Presentation: 3.4\%} \\
	We introduce GNeRF, a method that can estimate neural radiance fields and camera poses jointly when the cameras are initialized at random poses in complex scenarios (outside-in scenes, even with less texture or intense noise). We achieve this by marrying Generative Adversarial Networks (GAN) with Neural Radiance Field.
	
	\item {\bf Quan Meng}, Jiakai Zhang, Qiang Hu, Xuming He, and Jingyi Yu \\
	LGNN: A Context-Aware Line Segment Detector \\
	\textit{Proceedings of the 28th ACM International Conference on Multimedia (ACM MM), 2020} \\
	{\bf Poster: 27.9\%} \\
	Existing approaches require a computationally expensive verification or postprocessing step. Our LGNN employs a deep convolutional neural network (DCNN) for proposing line segments directly, with a graph neural network (GNN) module for reasoning their connectivities. LGNN achieves comparable performance and enables time-sensitive 3D applications. 
\end{enumerate}  

%% ========= Honors and Awards ===========================
\section{Honors and Awards}
\vspace{-0.25in}
\begin{outerlist}
\item First prize in \href{http://www.worldrobotconference.com/en/}{World Robot Contest Fighting Robot Competition}. \hfill 2017
\item First prize in \href{https://www.nuedc-training.com.cn/}{National Undergraduate Electronics Design Contest (Shandong, China)} \hfill 2017
\item First prize in \href{https://www.nuedc-training.com.cn/}{National Undergraduate Electronics Design Contest (Shandong, China)} \hfill 2018
\item Second prize in \href{http://edu.shandong.gov.cn/col/col12041/index.html}{The 4th Shandong College Students' Science and Innovation Contest} \hfill 2018
\item National Scholarship Award, ShanghaiTech University	 \hfill 2021

\end{outerlist}

%% ========= Teaching Experience ============================
\section{Teaching Experience}
\label{sec:teaching_exp}
\vspace{-0.25in}
\begin{outerlist}
	\item {\it CS280 Deep Learning} in ShanghaiTech University: Teaching Assistant \hfill Fall 2020
\end{outerlist}

%% ========= Technical Skills ============================
\section{Technical Skills} % (fold)
\label{sec:technical_ski}
\vspace{-0.25in}
\begin{outerlist}
	\item Programming: Linux, Assembly Language (ARM), C/C++, Python, Java, Pytorch, Opencv, Latex, Matlab, Qt.
	\item Hardwares: Circuit Design, ARM, STM32, STC, PCB. 
	\item Softwares: Blender, Keil, Illustrator
\end{outerlist}

%% ========= References ==============================
\section{References}
\label{sec:technical_ski}
\vspace{-0.25in}
\begin{outerlist}
	\item \textbf{Prof. Jingyi Yu}, ShanghaiTech University, \href{mailto:yujingyi@shanghaitech.edu.cn}{yujingyi@shanghaitech.edu.cn}
	\item \textbf{Prof. Hao Su}, UC San Diego, \href{mailto:haosu@eng.ucsd.edu}{haosu@eng.ucsd.edu}
	\item \textbf{Prof. Xuming He}, ShanghaiTech University, \href{mailto:hexm@shanghaitech.edu.cn}{hexm@shanghaitech.edu.cn}
	\item \textbf{Prof. Lan Xu}, ShanghaiTech University, \href{mailto:xulan1@shanghaitech.edu.cn}{xulan1@shanghaitech.edu.cn}
\end{outerlist}

%% ==============================================================
%\vspace{0.2in}
%\section{Research Background} % (fold)
%\label{sec:research_backg}
%\vspace{-0.25in}
%\begin{outerlist}
%	\item {\bf Social Network}: more descriptions here.
%	\item {\bf Advanced Vietnamese:} descriptions.
%\end{outerlist}
% section research_backg (end)
%% =========  ==============================
%\section{Research Experience} % (fold)
%\label{sec:research_exper}
%\vspace{-0.25in}
%\begin{outerlist}
%	\item {\bf Embedded Systems and Reconfigurable Computing Laboratory } \hfill 2009--2012\\
%	School of Electronics and Communications, Hanoi University of Science and Technology.\\
%	Implementations for the project below were done in a mix of VHDL/Verilog and C/C$++$.
%	\begin{innerlist}
%		\item FPGA-based Intellectual Property camera systems: Applications in remote supervising and controlling systems. The whole system is a combination of an ARM-based Board and a Xilinx platform. 
%	\end{innerlist}
%\end{outerlist}

\end{document}